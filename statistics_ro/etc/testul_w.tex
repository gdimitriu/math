
\documentclass{article}
%%%%%%%%%%%%%%%%%%%%%%%%%%%%%%%%%%%%%%%%%%%%%%%%%%%%%%%%%%%%%%%%%%%%%%%%%%%%%%%%%%%%%%%%%%%%%%%%%%%%%%%%%%%%%%%%%%%%%%%%%%%%%%%%%%%%%%%%%%%%%%%%%%%%%%%%%%%%%%%%%%%%%%%%%%%%%%%%%%%%%%%%%%%%%%%%%%%%%%%%%%%%%%%%%%%%%%%%%%%%%%%%%%%%%%%%%%%%%%%%%%%%%%%%%%%%
%TCIDATA{OutputFilter=LATEX.DLL}
%TCIDATA{Version=5.00.0.2552}
%TCIDATA{<META NAME="SaveForMode" CONTENT="1">}
%TCIDATA{Created=Wednesday, June 08, 2005 12:39:00}
%TCIDATA{LastRevised=Wednesday, June 08, 2005 18:29:51}
%TCIDATA{<META NAME="GraphicsSave" CONTENT="32">}
%TCIDATA{<META NAME="DocumentShell" CONTENT="Standard LaTeX\Blank - Standard LaTeX Article">}
%TCIDATA{CSTFile=40 LaTeX article.cst}

\setlength{\textheight}{8in}
\setlength{\textwidth}{6in}
\setlength{\oddsidemargin}{0mm}
\setlength{\evensidemargin}{0mm}
\setlength{\marginparwidth}{0mm}
\setlength{\marginparsep}{0mm}
\newtheorem{theorem}{Teorema}
\newtheorem{acknowledgement}[theorem]{Acknowledgement}
\newtheorem{algorithm}[theorem]{Algorithm}
\newtheorem{axiom}[theorem]{Axioma}
\newtheorem{case}[theorem]{Case}
\newtheorem{claim}[theorem]{Claim}
\newtheorem{conclusion}[theorem]{Concluzie}
\newtheorem{condition}[theorem]{Condition}
\newtheorem{conjecture}[theorem]{Conjecture}
\newtheorem{corollary}[theorem]{Corolar}
\newtheorem{criterion}[theorem]{Criteriu}
\newtheorem{definition}[theorem]{Definitie}
\newtheorem{example}[theorem]{Exemplu}
\newtheorem{exercise}[theorem]{Exercitiu}
\newtheorem{lemma}[theorem]{Lemma}
\newtheorem{notation}[theorem]{Notatie}
\newtheorem{problem}[theorem]{Problema}
\newtheorem{proposition}[theorem]{Propozitie}
\newtheorem{remark}[theorem]{Remarka}
\newtheorem{solution}[theorem]{Solutie}
\newtheorem{summary}[theorem]{Summary}
\newenvironment{proof}[1][Proof]{\noindent\textbf{Demostratie} }{\ \rule{0.5em}{0.5em}}
\input{tcilatex}

\begin{document}

\title{Testul W}
\author{Copyright 2005 Gabriel Dimitriu}
\maketitle

\begin{exercise}
Folosind testul W si pentru $\alpha =0.05$ sa se verifice ipoteza $%
H_{0}:m_{e}=50;H_{1}:m_{e}>50$ cu ajutorul datelor: 57, 45.5, 46, 52.5, 53,
58, 48, 71, 52, 57.5, 58, 44, 35.5, 54, 65.5.
\end{exercise}

\begin{proof}
Deoarece folosim testul W vom face tabelul:

\begin{tabular}{llllll}
$i$ & $x_{i}$ & $x_{i}-50$ & $\left\vert x_{i}-50\right\vert $ & $R_{i}$ & $%
R_{i}^{s}$ \\ 
1 & 57 & 7 & 7 & 9 & 9 \\ 
2 & 45.5 & -4.5 & 4.5 & 7 & -7 \\ 
3 & 46 & -4 & 4 & 5.5 & -5.5 \\ 
4 & 52.5 & 2.5 & 2.5 & 3 & 3 \\ 
5 & 53 & 3 & 3 & 4 & 4 \\ 
6 & 58 & 8 & 8 & 11.5 & 11.5 \\ 
7 & 48 & -2 & 2 & 1.5 & -1.5 \\ 
8 & 71 & 21 & 21 & 15 & 15 \\ 
9 & 52 & 2 & 2 & 1.5 & 1.5 \\ 
10 & 57.5 & 7.5 & 7.5 & 10 & 10 \\ 
11 & 58 & 8 & 8 & 11.5 & 11.5 \\ 
12 & 44 & -6 & 6 & 8 & -8 \\ 
13 & 35.5 & -14.5 & 14.5 & 13 & -13 \\ 
14 & 54 & 4 & 4 & 5.5 & 5.5 \\ 
15 & 65.5 & 15.5 & 15.5 & 14 & 14%
\end{tabular}

Calculez%
\[
w=\sum_{i=1}^{n}R_{i}^{s}=50
\]

Calculam%
\[
Z_{calc}=\frac{w}{\sqrt{\frac{(n+1)(n+2)(2n+1)}{6}}}=\frac{50}{\sqrt{16\cdot
17\cdot 31/6}}=\frac{50}{37.48}=1.33
\]

Din tabel luam $Z_{1-\alpha }=Z_{0.95}=1.65$.

Deoarece $Z_{calc}=1.33<Z_{1-\alpha }=1.65$ vom accepta ipoteza $H_{0}$.
\end{proof}

\begin{exercise}
Folosind testul W si pentru $\alpha =0.05$ sa se verifice ipoteaza $%
H_{0}:m_{e}=1.14;H_{1}:m_{e}>1.14$ utilizind urmatorul set de date: 1.2,
1.18, 1.25, 1.29, 1.12 , 1.11, 1.19, 1.23, 1.31, 1.20, 1.13, 1.23, 1.06,
1.17.
\end{exercise}

\begin{proof}
Deorece folosim testul W vom face urmatorul tabel:

\begin{tabular}{llllll}
$i$ & $x_{i}$ & $x_{i}-1.14$ & $\left\vert x_{i}-1.14\right\vert $ & $R_{i}$
& $R_{i}^{s}$ \\ 
1 & 1.2 & 0.06 & 0.06 & 7.5 & 7.5 \\ 
2 & 1.18 & 0.04 & 0.04 & 5 & 5 \\ 
3 & 1.25 & 0.11 & 0.11 & 12 & 12 \\ 
4 & 1.29 & 0.15 & 0.15 & 13 & 13 \\ 
5 & 1.12 & -0.02 & 0.02 & 2 & -2 \\ 
6 & 1.11 & -0.03 & 0.03 & 3.5 & -3.5 \\ 
7 & 1.19 & 0.05 & 0.05 & 6 & 6 \\ 
8 & 1.23 & 0.09 & 0.09 & 10.5 & 10.5 \\ 
9 & 1.31 & 0.17 & 0.17 & 14 & 14 \\ 
10 & 1.2 & 0.06 & 0.06 & 7.5 & 7.5 \\ 
11 & 1.13 & -0.01 & 0.01 & 1 & -1 \\ 
12 & 1.23 & 0.09 & 0.09 & 10.5 & 10.5 \\ 
13 & 1.06 & -0.08 & 0.08 & 9 & -9 \\ 
14 & 1.17 & 0.03 & 0.03 & 3.5 & 3.5%
\end{tabular}

Calculam%
\[
w=\sum_{i=1}^{n}R_{i}^{s}=74
\]

Acum calculam 
\[
Z_{calc}=\frac{w}{\sqrt{\frac{(n+1)(n+2)(2n+1)}{6}}}=\frac{74}{\sqrt{\frac{%
15\cdot 16\cdot 29}{6}}}=2.17
\]

Din tabel avem $Z_{1-\alpha }=Z_{1-0.05}=Z_{0.95}=1.65$.

Deoarece $Z_{calc}=2.17>Z_{1-\alpha }=1.65$ vom respinge ipoteza $H_{0}$.
\end{proof}

\begin{exercise}
Folosind testul W si pentru $\alpha =0.05$ sa se verifice ipotezele $%
H_{0}:m_{e}=22;H_{1}:m_{e}>22$ pentru urmatorul set de date: 21.5, 18.95,
19.4, 18.55, 19.15, 22.35, 22.9, 22.2, 23.1.
\end{exercise}

\begin{proof}
Deoarece vom folosi testul W vom face urmatorul tabel

\begin{tabular}{llllll}
$i$ & $x_{i}$ & $x_{i}-22$ & $\left\vert x_{i}-22\right\vert $ & $R_{i}$ & $%
R_{i}^{s}$ \\ 
1 & 21.5 & -0.5 & 0.5 & 3 & -3 \\ 
2 & 18.95 & -3.05 & 3.05 & 8 & -8 \\ 
3 & 19.4 & -2.6 & 2.6 & 6 & -6 \\ 
4 & 18.55 & -3.45 & 3.45 & 9 & -9 \\ 
5 & 19.15 & -2.85 & 2.85 & 7 & -7 \\ 
6 & 22.35 & 0.35 & 0.35 & 2 & 2 \\ 
7 & 22.9 & 0.9 & 0.9 & 4 & 4 \\ 
8 & 22.2 & 0.2 & 0.2 & 1 & 1 \\ 
9 & 23.1 & 1.1 & 1.1 & 5 & 5%
\end{tabular}

Calculam%
\[
w=\sum_{i=1}^{n}R_{i}^{s}=-21
\]

Calculam%
\[
Z_{calc}=\frac{w}{\sqrt{\frac{(n+1)(n+2)(2n+1)}{6}}}=\frac{-21}{\sqrt{%
10\cdot 11\cdot 21/6}}=-1.07
\]

Din tabel avem $Z_{1-\alpha }=Z_{1-0.05}=Z_{0.95}=1.65$.

Deoarece $Z_{calc}=-1.07<Z_{1-\alpha }=1.65$ vom accepta ipoteza $H_{0}.$
\end{proof}

\begin{exercise}
Folosind testul W pentru $\alpha =0.05$ sa se verifice ipotezele $%
H_{0}:m_{e}=7.20;H_{1}:m_{e}>7.20$ pentru urmatorul set de date: 8.25, 9.16,
9.55, 7, 6.55, 8.72, 6.7, 7.05, 7.2, 6.94, 7.16, 9.55, 5.98, 6.18, 7.82,
8.36.
\end{exercise}

\begin{proof}
Deorece utilizam testul W vom face urmatorul tabel

\begin{tabular}{llllll}
$i$ & $x_{i}$ & $x_{i}-7.2$ & $\left\vert x_{i}-7.2\right\vert $ & $R_{i}$ & 
$R_{i}^{s}$ \\ 
1 & 8.25 & 1.05 & 1.05 & 9 & 9 \\ 
2 & 9.16 & 1.96 & 1.96 & 13 & 13 \\ 
3 & 9.55 & 2.35 & 2.35 & 14.5 & 14.5 \\ 
4 & 7 & -0.2 & 0.2 & 3 & -3 \\ 
5 & 6.55 & -0.65 & 0.65 & 7 & -7 \\ 
6 & 8.72 & 1.52 & 1.52 & 12 & 12 \\ 
7 & 6.7 & -0.5 & 0.5 & 5 & -5 \\ 
8 & 7.05 & -0.15 & 0.15 & 2 & -2 \\ 
9 & 7.2 & 0 & 0 & 0 & 0 \\ 
10 & 6.94 & -0.26 & 0.26 & 4 & -4 \\ 
11 & 7.16 & -0.04 & 0.04 & 1 & -1 \\ 
12 & 9.55 & 2.35 & 2.35 & 14.5 & 14.5 \\ 
13 & 5.98 & -1.22 & 1.22 & 11 & -11 \\ 
14 & 6.18 & -1.02 & 1.02 & 8 & -8 \\ 
15 & 7.82 & 0.62 & 0.62 & 6 & 6 \\ 
16 & 8.36 & 1.16 & 1.16 & 10 & 10%
\end{tabular}

Calculam%
\[
w=\sum_{i=1}^{n}R_{i}^{s}=38
\]

Calculam de asemenea%
\[
Z_{calc}=\frac{w}{\sqrt{\frac{(n+1)(n+2)(2n+1)}{6}}}=\frac{38}{\sqrt{17\cdot
18\cdot 33/6}}=0.93
\]

Din tabel avem avem $Z_{1-\alpha }=Z_{1-0.05}=Z_{0.95}=1.65$.

Deoarece $Z_{calc}=0.93<Z_{1-\alpha }=1.65$ vom accepta ipoteza $H_{0}$.
\end{proof}

\begin{exercise}
Fie $m_{e}$ mediana diferentei (x-y) si vom verifica su ajutorul testului w
ipotezele in cazul $\alpha =0.05$ $H_{0}:m_{e}=0;H_{1}:m_{e}>0$ pentur
urmatorul set de date: (17.1,19.6), (14.6,12.3), (17.3,19.4), (18.5,14.3),
(13.5,18.6), (13.7,14.7), (15.4,15.4), (18.4,16.3), (16.6,16.2), (9.1,12.5),
(11.5,8.1), (7.5,8.4), (19.6,17.5), (6.4,9.1), (16.5,18.6), (19.8,17.7),
(14.3,13.2).
\end{exercise}

\begin{proof}
Deoarece utilizam testul W vom face urmatorul tabel

\begin{tabular}{lllllll}
$i$ & $x_{i}$ & $y_{i}$ & $x_{i}-y_{i}$ & $\left\vert
x_{i}-y_{i}-0\right\vert $ & $R_{i}$ & $R_{i}^{s}$ \\ 
1 & 17.1 & 19.6 & -2.5 & 2.5 & 11 & -11 \\ 
2 & 14.6 & 12.3 & 2.3 & 2.3 & 10 & 10 \\ 
3 & 17.3 & 19.4 & -2.1 & 2.1 & 7 & -7 \\ 
4 & 18.5 & 14.3 & 4.2 & 4.2 & 15 & 15 \\ 
5 & 13.5 & 18.6 & -5.1 & 5.1 & 16 & 16 \\ 
6 & 13.7 & 14.7 & -1 & 1 & 3 & -3 \\ 
7 & 15.4 & 15.4 & 0 & 0 & 0 & 0 \\ 
8 & 18.4 & 16.3 & 2.1 & 2.1 & 7 & 7 \\ 
9 & 16.6 & 16.2 & 0.4 & 0.4 & 1 & 1 \\ 
10 & 9.1 & 12.5 & -3.4 & 3.4 & 13.5 & -13.5 \\ 
11 & 11.5 & 8.1 & 3.4 & 3.4 & 13.5 & 13.5 \\ 
12 & 7.5 & 8.4 & -0.9 & 0.9 & 2 & -2 \\ 
13 & 19.6 & 17.5 & 2.1 & 2.1 & 7 & 7 \\ 
14 & 6.4 & 9.1 & -2.7 & 2.7 & 12 & -12 \\ 
15 & 16.5 & 18.6 & -2.1 & 2.1 & 7 & -7 \\ 
16 & 19.8 & 17.7 & 2.1 & 2.1 & 7 & 7 \\ 
17 & 14.3 & 13.2 & 1.1 & 1.1 & 4 & 4%
\end{tabular}

Calculam%
\[
w=\sum_{i=1}^{n}R_{i}^{s}=25
\]

Calculam de asemenea%
\[
Z_{calc}=\frac{w}{\sqrt{\frac{(n+1)(n+2)(2n+1)}{6}}}=\frac{25}{\sqrt{18\cdot
19\cdot 35/6}}=0.56
\]

Din tabel avem $Z_{1-\alpha }=Z_{1-0.05}=Z_{0.95}=1.65$.

Deoarece $Z_{calc}=0.56<Z_{1-\alpha }=1.65$ vom accepta ipoteza $H_{0}$.
\end{proof}

\begin{exercise}
Fie $m_{e}$ mediatoarea diferentei x-y sa se verifice utilizind testul W la
pragul $\alpha =0.05$ ipotezele $H_{0}:m_{e}=0;H_{1}:m_{e}>0$ pentru
urmatorul set de date: (8.32,7.80), (6.45,7), (5.95,6.2), (7.36,8.14),
(7.83,8.33), (7.30,8.15), (5.4,5.55), (6.9,8.1), (7,4.8), (8.17,7.77),
(6.68,7.18), (6.95,8.45), (7.26,8.56), (15.43,7.28), (8.14,6.44).
\end{exercise}

\begin{proof}
Deoarece folosim testul vem vom realiza tabelul

\begin{tabular}{lllllll}
$i$ & $x_{i}$ & $y_{i}$ & $x_{i}-y_{i}$ & $\left\vert
x_{i}-y_{i}-0\right\vert $ & $R_{i}$ & $R_{i}^{s}$ \\ 
1 & 8.32 & 7.8 & 0.52 & 0.52 & 6 & 6 \\ 
2 & 6.45 & 7 & -0.55 & 0.55 & 7 & -7 \\ 
3 & 5.95 & 6.2 & -0.25 & 0.25 & 2 & -2 \\ 
4 & 7.36 & 8.14 & -0.78 & 0.78 & 8 & -8 \\ 
5 & 7.83 & 8.33 & -0.5 & 0.5 & 4.5 & -4.5 \\ 
6 & 7.3 & 8.15 & -0.85 & 0.85 & 9 & -9 \\ 
7 & 5.4 & 5.55 & -0.15 & 0.15 & 1 & -1 \\ 
8 & 6.9 & 8.1 & -1.2 & 1.2 & 10 & -10 \\ 
9 & 7 & 4.8 & 2.2 & 2.2 & 14 & 14 \\ 
10 & 8.17 & 7.77 & 0.4 & 0.4 & 3 & 3 \\ 
11 & 6.68 & 7.18 & -0.5 & 0.5 & 4.5 & -4.5 \\ 
12 & 6.95 & 8.45 & -1.5 & 1.5 & 12 & -12 \\ 
13 & 7.26 & 8.56 & -1.3 & 1.3 & 11 & -11 \\ 
14 & 15.43 & 7.28 & 8.15 & 8.15 & 15 & 15 \\ 
15 & 8.14 & 6.44 & 1.7 & 1.7 & 13 & 13%
\end{tabular}

Calculam%
\[
w=\sum_{i=1}^{n}R_{i}^{s}=-18
\]

Calculam de asemenea%
\[
Z_{calc}=\frac{w}{\sqrt{\frac{(n+1)(n+2)(2n+1)}{6}}}=\frac{-18}{\sqrt{%
16\cdot 17\cdot 31/6}}=-0.48
\]

Din tabel avem $Z_{1-\alpha }=Z_{1-0.05}=Z_{0.95}=1.65$.

Deoarece $Z_{calc}=-0.48<Z_{1-\alpha }=1.65$ vom accepta ipoteza $H_{0}$.
\end{proof}

\end{document}
