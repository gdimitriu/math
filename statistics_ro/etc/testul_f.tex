
\documentclass{article}
%%%%%%%%%%%%%%%%%%%%%%%%%%%%%%%%%%%%%%%%%%%%%%%%%%%%%%%%%%%%%%%%%%%%%%%%%%%%%%%%%%%%%%%%%%%%%%%%%%%%%%%%%%%%%%%%%%%%%%%%%%%%%%%%%%%%%%%%%%%%%%%%%%%%%%%%%%%%%%%%%%%%%%%%%%%%%%%%%%%%%%%%%%%%%%%%%%%%%%%%%%%%%%%%%%%%%%%%%%%%%%%%%%%%%%%%%%%%%%%%%%%%%%%%%%%%
%TCIDATA{OutputFilter=LATEX.DLL}
%TCIDATA{Version=5.00.0.2552}
%TCIDATA{<META NAME="SaveForMode" CONTENT="1">}
%TCIDATA{Created=Sunday, June 05, 2005 11:18:33}
%TCIDATA{LastRevised=Sunday, June 05, 2005 11:19:28}
%TCIDATA{<META NAME="GraphicsSave" CONTENT="32">}
%TCIDATA{<META NAME="DocumentShell" CONTENT="Standard LaTeX\Blank - Standard LaTeX Article">}
%TCIDATA{CSTFile=40 LaTeX article.cst}
\setlength{\textheight}{8in}
\setlength{\textwidth}{6in}
\setlength{\oddsidemargin}{0mm}
\setlength{\evensidemargin}{0mm}
\setlength{\marginparwidth}{0mm}
\setlength{\marginparsep}{0mm}
\newtheorem{theorem}{Teorema}
\newtheorem{acknowledgement}[theorem]{Acknowledgement}
\newtheorem{algorithm}[theorem]{Algorithm}
\newtheorem{axiom}[theorem]{Axioma}
\newtheorem{case}[theorem]{Case}
\newtheorem{claim}[theorem]{Claim}
\newtheorem{conclusion}[theorem]{Concluzie}
\newtheorem{condition}[theorem]{Condition}
\newtheorem{conjecture}[theorem]{Conjecture}
\newtheorem{corollary}[theorem]{Corolar}
\newtheorem{criterion}[theorem]{Criteriu}
\newtheorem{definition}[theorem]{Definitie}
\newtheorem{example}[theorem]{Exemplu}
\newtheorem{exercise}[theorem]{Exercitiu}
\newtheorem{lemma}[theorem]{Lemma}
\newtheorem{notation}[theorem]{Notatie}
\newtheorem{problem}[theorem]{Problema}
\newtheorem{proposition}[theorem]{Propositie}
\newtheorem{remark}[theorem]{Remarka}
\newtheorem{solution}[theorem]{Solutie}
\newtheorem{summary}[theorem]{Summary}
\newenvironment{proof}[1][Proof]{\noindent\textbf{Demostratie} }{\ \rule{0.5em}{0.5em}}
\input{tcilatex}

\begin{document}

\title{Testul F}
\author{Copyright 2005 Gabriel Dimitriu}
\maketitle

\begin{exercise}
\bigskip
\end{exercise}

O selectie de volum $n_{1}=10$ din becurile avind marca 1 a dat $\overline{x}%
_{10}=932ore$ si $s_{10}^{2}=8114$. O selectie de volum $n_{2}=14$ din
becurile avind marca 2 a dat $\overline{x}_{14}=859ore$ si $s_{14}^{2}=5324$%
. Sa se foloseasca aceste date pentru a verifica ipoteza $H_{0}:\sigma
_{1}^{2}=\sigma _{2}^{2}$ fata de $H_{1}:\sigma _{1}^{2}>\sigma _{2}^{2}$ la
pragul de semnificatie $\alpha =0.01$.

\textbf{Rezolvare:}

Deoarece avem de verificat egalitata a doua dispersii vom folosi testul
Fisher. Deoarece avem \TEXTsymbol{>} in ipoteza $H_{1}$ vom folosi testul
Fisher unilateral dreapta.

Pentru aceasta calculam:

\[
\frac{s_{10}^{2}}{s_{14}^{2}}=\frac{8114}{5324}=1.524 
\]

Cautam in tabel quantila si avem

\[
F_{1-\alpha }(n_{2}-1,n_{1}-1)=F_{0.99}(13,9)=5.11 
\]

Deoarece $\frac{s_{10}^{2}}{s_{14}^{2}}=1.524<F_{0.99}(13,9)=5.11$ vom
respinge ipoteza $H_{0}$.

\end{document}
