
\documentclass{article}
%%%%%%%%%%%%%%%%%%%%%%%%%%%%%%%%%%%%%%%%%%%%%%%%%%%%%%%%%%%%%%%%%%%%%%%%%%%%%%%%%%%%%%%%%%%%%%%%%%%%%%%%%%%%%%%%%%%%%%%%%%%%%%%%%%%%%%%%%%%%%%%%%%%%%%%%%%%%%%%%%%%%%%%%%%%%%%%%%%%%%%%%%%%%%%%%%%%%%%%%%%%%%%%%%%%%%%%%%%%%%%%%%%%%%%%%%%%%%%%%%%%%%%%%%%%%
%TCIDATA{OutputFilter=LATEX.DLL}
%TCIDATA{Version=5.00.0.2552}
%TCIDATA{<META NAME="SaveForMode" CONTENT="1">}
%TCIDATA{Created=Tuesday, March 22, 2005 18:05:35}
%TCIDATA{LastRevised=Tuesday, March 22, 2005 19:03:31}
%TCIDATA{<META NAME="GraphicsSave" CONTENT="32">}
%TCIDATA{<META NAME="DocumentShell" CONTENT="Standard LaTeX\Blank - Standard LaTeX Article">}
%TCIDATA{CSTFile=40 LaTeX article.cst}

\setlength{\textheight}{8in}
\setlength{\textwidth}{6in}
\setlength{\oddsidemargin}{0mm}
\setlength{\evensidemargin}{0mm}
\setlength{\marginparwidth}{0mm}
\setlength{\marginparsep}{0mm}
\newtheorem{theorem}{Theorem}
\newtheorem{acknowledgement}[theorem]{Acknowledgement}
\newtheorem{algorithm}[theorem]{Algorithm}
\newtheorem{axiom}[theorem]{Axiom}
\newtheorem{case}[theorem]{Case}
\newtheorem{claim}[theorem]{Claim}
\newtheorem{conclusion}[theorem]{Conclusion}
\newtheorem{condition}[theorem]{Condition}
\newtheorem{conjecture}[theorem]{Conjecture}
\newtheorem{corollary}[theorem]{Corollary}
\newtheorem{criterion}[theorem]{Criterion}
\newtheorem{definition}[theorem]{Definition}
\newtheorem{example}[theorem]{Example}
\newtheorem{exercise}[theorem]{Exercise}
\newtheorem{lemma}[theorem]{Lemma}
\newtheorem{notation}[theorem]{Notation}
\newtheorem{problem}[theorem]{Problem}
\newtheorem{proposition}[theorem]{Proposition}
\newtheorem{remark}[theorem]{Remark}
\newtheorem{solution}[theorem]{Solution}
\newtheorem{summary}[theorem]{Summary}
\newenvironment{proof}[1][Proof]{\noindent\textbf{#1.} }{\ \rule{0.5em}{0.5em}}
\input{tcilatex}

\begin{document}

\title{Teste statistice}
\author{Dimitriu Gabriel Spiru Haret AnIII}
\maketitle

\begin{exercise}
\bigskip 
\end{exercise}

Greutatea unor pachete de zahar pudra ca urmare a imperfectiunii procesului
de impachetare este o variabila aleatoare $N(m,5^{2})$. Greutatea marcata pe
pachete este 900 g. O selectie aleatoare de volum 10 conduce la o greutate
medie observata de 898 g. Sa se testeze la prag de semnificatie 0.05 ipoteza
procesul de impachetare \ este bine reglat. Sa se calculeze probabilitatea
acceptarii ipotezei H$_{0}:m=m_{0}=900\,g$ pentru $m_{1}=898g.$

\textbf{Rezolvare:}

Din enuntul problemei avem estimatorul mediei 

\[
\overline{x}=m_{1}=898g
\]

pragul de semnificatie 
\[
\alpha =0.05
\]

dispersia procesului este

\[
\sigma =5
\]

numarul elementelor testare

\[
n=10
\]

Vom testa cele doua ipoteze statistice pentru a vedea daca procesul este
bine reglat

\begin{eqnarray*}
H_{0} &:&m=m_{0}=900g \\
H_{1} &:&m\neq m_{0}=900g
\end{eqnarray*}

Deoarece $m\neq m_{0}$ vom aplica testul bilateral pentru populatie normala
de medie necunoscuta si dispersie cunoscuta.

Testul consta in a verifica daca

\begin{eqnarray*}
\left\vert \frac{\overline{x}-m_{0}}{\sigma /\sqrt{n}}\right\vert 
&>&u_{1-\alpha /2}\,:\,resping\,\ H_{0} \\
\left\vert \frac{\overline{x}-m_{0}}{\sigma /\sqrt{n}}\right\vert 
&<&u_{1-\alpha /2}\,:\,accept\ H_{0}
\end{eqnarray*}

Facem calculul

\[
\left\vert \frac{\overline{x}-m_{0}}{\sigma /\sqrt{n}}\right\vert
=\left\vert \frac{898-900}{5/3.2622}\right\vert =\left\vert
-1.2649\right\vert =1.2649
\]

Din tabel avem

\[
u_{1-\frac{\alpha }{2}}=u_{0.975}=1.96
\]

Deoarece $1.96>1.2649$ vom accepta ipoteza $H_{0}$ deci procesul este bine
reglat cu pragul de semnificatie $\alpha =0.05$.

Probabilitatea acceptarii ipotezei $H_{0}:m=m_{0}=900g$ in cazul in care a
fost $m_{1}=898g$ este data de puterea testului data prin formula:

\[
\Pi (m_{1})=\Pi (898)=P(x\in W|m=m_{1})=\Phi (\frac{m_{1}-m_{0}}{\sigma /%
\sqrt{n}}-u_{1-\alpha /2})
\]

inlocuind avem

\[
\Pi (m_{1}=898)=\Phi (1.2649-1.96)=\Phi (-0.6951)=0.2435
\]

\end{document}
