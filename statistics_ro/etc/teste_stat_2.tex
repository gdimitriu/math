
\documentclass{article}
%%%%%%%%%%%%%%%%%%%%%%%%%%%%%%%%%%%%%%%%%%%%%%%%%%%%%%%%%%%%%%%%%%%%%%%%%%%%%%%%%%%%%%%%%%%%%%%%%%%%%%%%%%%%%%%%%%%%%%%%%%%%%%%%%%%%%%%%%%%%%%%%%%%%%%%%%%%%%%%%%%%%%%%%%%%%%%%%%%%%%%%%%%%%%%%%%%%%%%%%%%%%%%%%%%%%%%%%%%%%%%%%%%%%%%%%%%%%%%%%%%%%%%%%%%%%
%TCIDATA{OutputFilter=LATEX.DLL}
%TCIDATA{Version=5.00.0.2552}
%TCIDATA{<META NAME="SaveForMode" CONTENT="1">}
%TCIDATA{Created=Tuesday, April 19, 2005 20:02:55}
%TCIDATA{LastRevised=Monday, January 23, 2006 16:35:43}
%TCIDATA{<META NAME="GraphicsSave" CONTENT="32">}
%TCIDATA{<META NAME="DocumentShell" CONTENT="Standard LaTeX\Blank - Standard LaTeX Article">}
%TCIDATA{Language=American English}
%TCIDATA{CSTFile=40 LaTeX article.cst}

\setlength{\textheight}{8in}
\setlength{\textwidth}{6in}
\setlength{\oddsidemargin}{0mm}
\setlength{\evensidemargin}{0mm}
\setlength{\marginparwidth}{0mm}
\setlength{\marginparsep}{0mm}
\newtheorem{theorem}{Teorema}
\newtheorem{acknowledgement}[theorem]{Acknowledgement}
\newtheorem{algorithm}[theorem]{Algorithm}
\newtheorem{axiom}[theorem]{Axioma}
\newtheorem{case}[theorem]{Case}
\newtheorem{claim}[theorem]{Claim}
\newtheorem{conclusion}[theorem]{Concluzie}
\newtheorem{condition}[theorem]{Condition}
\newtheorem{conjecture}[theorem]{Conjecture}
\newtheorem{corollary}[theorem]{Corolar}
\newtheorem{criterion}[theorem]{Criteriu}
\newtheorem{definition}[theorem]{Definitie}
\newtheorem{example}[theorem]{Exemplu}
\newtheorem{exercise}[theorem]{Exercitiu}
\newtheorem{lemma}[theorem]{Lemma}
\newtheorem{notation}[theorem]{Notatie}
\newtheorem{problem}[theorem]{Problema}
\newtheorem{proposition}[theorem]{Propozitie}
\newtheorem{remark}[theorem]{Remarka}
\newtheorem{solution}[theorem]{Solutie}
\newtheorem{summary}[theorem]{Summary}
\newenvironment{proof}[1][Proof]{\noindent\textbf{#1.} }{\ \rule{0.5em}{0.5em}}
\input{tcilatex}

\begin{document}

\title{Teste statistice 2}
\author{Copyright Gabriel Dimitriu}
\maketitle

\begin{exercise}
\bigskip 
\end{exercise}

Nivelul de calciu in singele \ unui adult tinar este in medie 9.5mg/dl si cu 
$\sigma =0.4$. O clinica masoara nivelul calciului la 160 de pacienti si
gaseste media de 9.3. Verificati ipoteza: $H_{0}:\,m=9.5;H_{1}:m\neq
9.5;\alpha =0.05$.

\textbf{Rezolvare:}

Deoarece avem $H_{1}:\,m\neq 9.5$ vom considera un test bilitater si
deoarece avem cunoscuta media vom folosi testul Z.

Calculam

\[
Z_{calc}=\frac{\overline{x}-m_{0}}{\sigma /\sqrt{n}} 
\]

unde: $\overline{x}=9.3$,$\sigma =0.4$, $m_{0}=9.5$, $n=160$.

Inlocuind aceste valori in formula anterioara avem

\[
Z_{calc}=-6.32 
\]

Deoarece avem test bilateral avem

\[
Z_{tab}=Z_{1-\alpha /2}=Z_{1-0.025}=Z_{0.975}=1.96 
\]

Pentru a verifica corectitudinea ipotezelor vom efectua testul:

\[
\left\vert Z_{calc}\right\vert =6.32>Z_{tab}=1.96 
\]

Deci vom respinge ipoteza nula $H_{0}$.

\begin{exercise}
\end{exercise}

Se iau esantioane din apa rezultata din racirea la o centrala nucleara. Se
considera ca daca temperatura apei evacuate nu depaseste 60$^{0}C$ nu
consituie o primejdie \ pentru mediul inconjurator. Se aleg 70 de esantionae
de apa si se masoara temperatura fiecarui asemenea esantion si se obtin
rezultatele:

\[
\begin{tabular}{lllllll}
temp(x) & 52 & 54 & 58 & 61 & 64 & 65 \\ 
frecv & 14 & 21 & 18 & 10 & 5 & 2%
\end{tabular}%
\]

a)Care sunt erorile de tip I si de tip II ce apar la verificarea ipotezei:

\begin{eqnarray*}
H_{0} &:&\,m=60 \\
H_{1} &:&m>60
\end{eqnarray*}

b)Aflati puterea testului folosit.

\textbf{Rezolvare:}

Vom calcula media si dispersia de selectie a esantioanelor:

\begin{eqnarray*}
\overline{x} &=&\frac{\sum_{i=1}^{6}x\ast frecv}{70}=56.657 \\
s^{2} &=&\frac{\sum_{i=1}^{6}frecv\ast (x_{i}-\overline{x})}{69}%
=15.677\rightarrow s=\sqrt{s^{2}}=\sqrt{15.677}=3.959
\end{eqnarray*}

Pentru a calcula erorile de tip I si II\ avem definitiile lor, in care $%
T_{n}(x_{1},x_{2},...,x_{n})$ este o statistica iar $W$ si $\overline{W}$
sunt doua regiuni complementare care vor defini erorile:

\begin{eqnarray*}
\alpha &=&P(T_{n}(x_{1},..,x_{n})\in W|H_{0})=P\left( \frac{\overline{x}-m}{%
s/\sqrt{n}}>t_{1-\alpha ;n-1}|m=m_{0}\right) \\
\beta &=&P(T_{n}(x_{1},..,x_{n})\in \overline{W}|H_{1})=P\left( \frac{%
\overline{x}-m}{s/\sqrt{n}}<t_{\alpha ;n-1}|m>m_{0}\right)
\end{eqnarray*}

\bigskip Inlocuind avem:

\[
\alpha =P(-7.064>t_{1-\alpha ;69})=\int_{-\infty }^{-7.0129}\frac{\Gamma (35)%
}{14.723\ast \Gamma (34.5)}(1+\frac{x^{2}}{69})^{-35}dx=0.6302526901\ast
10^{-9} 
\]

\end{document}
