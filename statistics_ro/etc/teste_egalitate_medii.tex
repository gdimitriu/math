
\documentclass{article}
%%%%%%%%%%%%%%%%%%%%%%%%%%%%%%%%%%%%%%%%%%%%%%%%%%%%%%%%%%%%%%%%%%%%%%%%%%%%%%%%%%%%%%%%%%%%%%%%%%%%%%%%%%%%%%%%%%%%%%%%%%%%%%%%%%%%%%%%%%%%%%%%%%%%%%%%%%%%%%%%%%%%%%%%%%%%%%%%%%%%%%%%%%%%%%%%%%%%%%%%%%%%%%%%%%%%%%%%%%%%%%%%%%%%%%%%%%%%%%%%%%%%%%%%%%%%
%TCIDATA{OutputFilter=LATEX.DLL}
%TCIDATA{Version=5.00.0.2552}
%TCIDATA{<META NAME="SaveForMode" CONTENT="1">}
%TCIDATA{Created=Tuesday, June 07, 2005 18:57:28}
%TCIDATA{LastRevised=Tuesday, June 07, 2005 20:23:14}
%TCIDATA{<META NAME="GraphicsSave" CONTENT="32">}
%TCIDATA{<META NAME="DocumentShell" CONTENT="Standard LaTeX\Blank - Standard LaTeX Article">}
%TCIDATA{CSTFile=40 LaTeX article.cst}

\setlength{\textheight}{8in}
\setlength{\textwidth}{6in}
\setlength{\oddsidemargin}{0mm}
\setlength{\evensidemargin}{0mm}
\setlength{\marginparwidth}{0mm}
\setlength{\marginparsep}{0mm}
\newtheorem{theorem}{Teorema}
\newtheorem{acknowledgement}[theorem]{Acknowledgement}
\newtheorem{algorithm}[theorem]{Algorithm}
\newtheorem{axiom}[theorem]{Axioma}
\newtheorem{case}[theorem]{Case}
\newtheorem{claim}[theorem]{Claim}
\newtheorem{conclusion}[theorem]{Concluzie}
\newtheorem{condition}[theorem]{Condition}
\newtheorem{conjecture}[theorem]{Conjecture}
\newtheorem{corollary}[theorem]{Corolar}
\newtheorem{criterion}[theorem]{Criteriu}
\newtheorem{definition}[theorem]{Definitie}
\newtheorem{example}[theorem]{Exemplu}
\newtheorem{exercise}[theorem]{Exercitiu}
\newtheorem{lemma}[theorem]{Lemma}
\newtheorem{notation}[theorem]{Notatie}
\newtheorem{problem}[theorem]{Problema}
\newtheorem{proposition}[theorem]{Propozitie}
\newtheorem{remark}[theorem]{Remarka}
\newtheorem{solution}[theorem]{Solutie}
\newtheorem{summary}[theorem]{Summary}
\newenvironment{proof}[1][Proof]{\noindent\textbf{Demonstratie} }{\ \rule{0.5em}{0.5em}}
\input{tcilatex}

\begin{document}

\title{Teste pentru egalitati de medii}
\author{Dimitriu Gabriel Spiru Haret AnIII}
\maketitle

\begin{exercise}
\bigskip 
\end{exercise}

Pentru a obtine informatii asupra rezistentei la coroziune a unui anumit tip
de conducta de otel se face un exepriment in care 40 de specimene sunt
ngropate timp de 3 ani si apoi masurind pentru fiecare specimen penetrarea
s-a obtinut: $\overline{x}=43.7,s=2.7$. Conducta este fabricata cu
specificatia ca penetratia este $m=46$. Pentru a vedea daca datele
experimentale indica faptul ca \ specificarea nu este corecta sa se verifice
la $\alpha =0.05$ $H_{0}:m=46;H_{1}:m>46$.

\begin{proof}
In cazul nostru avem urmatoarele date:%
\begin{eqnarray*}
n &=&40 \\
\overline{x} &=&43.7 \\
s &=&2.7 \\
m_{0} &=&46 \\
\alpha  &=&0.05
\end{eqnarray*}

Deoarece avem test de medie in care nu cunoastem dispersia vom utiliza
testul F unilateral dreapta.

Calculam dispersia de calcul

\[
s_{xx}^{2}=n(\overline{x}-m_{0})^{2}=40(43.7-46)^{2}=211.6
\]

Calculam $F_{calc}$%
\[
F_{calc}=\frac{s_{xx}^{2}}{s^{2}}=\frac{211.6}{(2.7)^{2}}=29
\]

Luam din tabele cuantiala%
\[
F_{1-\alpha }(1,n-1)=F_{1-0.05}(1,39)=F_{0.95}(1,39)=4.09
\]

Deoarece avem test unilateral dreapta in care%
\[
F_{calc}=29>F_{0.95}(1,39)=4.09
\]

vom respinge ipoteza $H_{0}$.
\end{proof}

\begin{exercise}
Dupa culegerea datelor a doua procese tehnologice avem urmatoarele masuratori
\end{exercise}

Pentru primul proces: $n_{1}=12$ proces de distributie $N(m_{1},20^{2})$:

\begin{tabular}{lllllllllllll}
$i$ & 1 & 2 & 3 & 4 & 5 & 6 & 7 & 8 & 9 & 10 & 11 & 12 \\ 
$x_{1,i}$ & 0.7 & 0.9 & 0.3 & 0.8 & 0.5 & 1 & 1.3 & 1.1 & 0.6 & 0.2 & 1.2 & 
1.4%
\end{tabular}

Pentru al doilea proces: $n_{2}=10$ proces de distributie $N(m_{2},25^{2})$:

\begin{tabular}{lllllllllll}
$i$ & 1 & 2 & 3 & 4 & 5 & 6 & 7 & 8 & 9 & 10 \\ 
$x_{2,i}$ & 0.9 & 0.8 & 1.4 & 1.3 & 0.7 & 1.9 & 1 & 1.6 & 1.2 & 2%
\end{tabular}

La pragul de incredere $\alpha =0.05$ sa se verifice ipoteza $%
H_{0}:m_{1}=m_{2}$ fata de contraipoteza $H_{1}:m_{1}<m_{2}$.

\begin{proof}
Deoarece avem un test de egalitate a mediilor cind cunoastem dispersia celor
doua procese vom aplica testul Z.

Asadar intii vom calcula mediile de selectie ale celor doua procese:

\begin{eqnarray*}
\overline{x_{n_{1}}} &=&\frac{1}{n_{1}}\sum_{i=1}^{n_{1}}x_{1,i}=0.83 \\
\overline{x_{n_{2}}} &=&\frac{1}{n_{2}}\sum_{i=1}^{n_{2}}x_{2,i}=1.12
\end{eqnarray*}

Acum calculam $Z_{calc}$%
\[
Z_{calc}=\frac{\overline{x_{n_{1}}}-\overline{x_{n_{2}}}+m_{1}-m_{2}}{\sqrt{%
\frac{\sigma _{1}^{2}}{n_{1}}+\frac{\sigma _{2}^{2}}{n_{2}}}}=\frac{%
0.83-1.12+0}{\sqrt{\frac{20^{2}}{12}+\frac{25^{2}}{10}}}=-0.029
\]

Din tabel obtine $Z_{\alpha }=Z_{0.05}=1.96$.

Cum $Z_{cal}=-0.029<Z_{\alpha }=1.96$ vom respinge ipoteza $H_{0}$.
\end{proof}

\begin{exercise}
In urma a $n=8$ masuratori asupra a doua procese tehnologice s-au obtinut
urmatoarele date
\end{exercise}

\begin{tabular}{lllllllll}
$i$ & 1 & 2 & 3 & 4 & 5 & 6 & 7 & 8 \\ 
$x_{1,i}$ & 10.28 & 10.27 & 10.3 & 10.32 & 10.27 & 10.27 & 10.28 & 10.29 \\ 
$x_{2,i}$ & 10.31 & 10.31 & 10.26 & 10.3 & 10.27 & 10.31 & 10.20 & 10.26%
\end{tabular}

Se se verifice la pragul de semnificatie $\alpha =0.01$ ca cele doua procese
au aceeasi medie.

\begin{proof}
Deoarece avem test asupra mediei a doua procese distincte si in acelasi timp
nu cunoastem dispersiile celor doua procese vom utiliza testul Student
pentru doua medii.

Ipotezele statistice sunt:

\begin{eqnarray*}
H_{0} &:&m_{1}=m_{2} \\
H_{1} &:&m_{1}\neq m_{2}
\end{eqnarray*}

Vom calcula mediile si dispersiile de selectie ale celor doua procese:%
\begin{eqnarray*}
\overline{x_{1}} &=&\frac{1}{n}\sum_{i=1}^{n}x_{1,i}=10.285 \\
\overline{x_{2}} &=&\frac{1}{n}\sum_{i=1}^{n}x_{2,i}=10.289 \\
s_{1}^{2} &=&\frac{1}{n-1}\sum_{i=1}^{n}(x_{1,i}-\overline{x_{1}}%
)^{2}=0.000314 \\
s_{2}^{2} &=&\frac{1}{n-1}\sum_{i=1}^{n}(x_{2,i}-\overline{x_{2}}%
)^{2}=0.000498
\end{eqnarray*}

Vom calcula statistica $t_{calc}$:%
\[
t_{calc}=\frac{\overline{x_{1}}-\overline{x_{2}}+m_{1}-m_{2}}{\sqrt{\frac{%
s_{1}^{2}}{n_{1}}+\frac{s_{2}^{2}}{n_{2}}}}=\frac{10.285-10.289}{\sqrt{\frac{%
0.000314}{8}+\frac{0.00498}{8}}}=0.398
\]

Din tabel luam cuantila $t_{\alpha
}(n+n-2)=t_{0.01}(8+8-2)=t_{0.01}(14)=-t_{0.99}(14)=-2.62$

Deoarece $\left\vert t_{calc}\right\vert =0.398>t_{0.01}(14)=-2.62$ vom
respinge ipoteza $H_{0}$.
\end{proof}

\begin{exercise}
\end{exercise}

Greutatea carnii amabalata in pachete de 1000g de masina M$_{1}$ este o
variabila aleatoare normala cu $\sigma _{1}=3g$ iar cea ambalata de masina M$%
_{2}$ are $\sigma _{2}=4g$. S-au cintarit 100 de pachete din produsele
fiecarei masini si-au obtinut $\overline{x_{1}}=1007g$ si $\overline{x_{2}}%
=1002g$. Sa se verifice $H_{0}:m_{1}=m_{2};H_{1}:m_{1}\neq m_{2}$ la $\alpha
=0.05$.

\begin{proof}
In conformitate cu enuntul problemei vom folosi testul Z pentru 2 medii si
avem:%
\begin{eqnarray*}
n_{1} &=&n_{2}=100 \\
\overline{x_{1}} &=&1007 \\
\overline{x_{2}} &=&1002 \\
\sigma _{1} &=&3 \\
\sigma _{2} &=&4
\end{eqnarray*}

Cu aceste date calculam%
\[
Z_{calc}=\frac{\overline{x_{1}}-\overline{x_{2}}-0}{\sqrt{\frac{\sigma
_{1}^{2}}{n_{1}}+\frac{\sigma _{2}^{2}}{n_{2}}}}=\frac{1007-1002}{\sqrt{%
\frac{9}{100}+\frac{16}{100}}}=10
\]

Luam din tabel $Z_{tab}=Z_{1-\alpha /2}=Z_{0.975}=1.9$.

Deoarece $\left\vert Z_{calc}\right\vert =10>Z_{tab}=1.9$ vom respinge
ipoteza $H_{0}$.
\end{proof}

\begin{exercise}
\end{exercise}

Dintr-o populatie normala cu $\sigma =5$ s-au extras doua selectii de volum
9. Selectiile au dat mediile de selectie 2 respectiv 3. Se poate afirma ca
la pragul de semnificatie de 0.05 ca diferentele inregistrate sunt
intimplatoare ?

\begin{proof}
Din problema avem urmatoarele date%
\begin{eqnarray*}
\sigma  &=&5 \\
n &=&9 \\
\alpha  &=&0.05 \\
\overline{x_{1}} &=&2 \\
\overline{x_{2}} &=&3 \\
H_{0} &:&m_{1}=m_{2} \\
H_{1} &:&m_{1}\neq m_{2}
\end{eqnarray*}

Deoarece cunoastem dispersiile vom utiliza testul Z pentru 2 medii.

Calculam $Z_{calc}$:%
\[
\left\vert Z_{calc}\right\vert =\left\vert \frac{\overline{x_{1}}-\overline{%
x_{2}}}{\sqrt{2\frac{\sigma ^{2}}{n}}}\right\vert =\left\vert \frac{-1}{%
\sqrt{\frac{2\cdot 25}{9}}}\right\vert =\frac{3}{5\sqrt{2}}=0.424
\]

Din tabel vom lua cuantila $Z_{tab}=Z_{1-\alpha /2}=Z_{0.975}=1.9$.

Deoarece $\left\vert Z_{calc}\right\vert =0.424<Z_{tab}=1.9$ vom accepta
ipoteza $H_{0}$, deci diferentele inregistrate sunt intimplatoare.
\end{proof}

\end{document}
