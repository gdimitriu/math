
\documentclass{article}
%%%%%%%%%%%%%%%%%%%%%%%%%%%%%%%%%%%%%%%%%%%%%%%%%%%%%%%%%%%%%%%%%%%%%%%%%%%%%%%%%%%%%%%%%%%%%%%%%%%%%%%%%%%%%%%%%%%%%%%%%%%%%%%%%%%%%%%%%%%%%%%%%%%%%%%%%%%%%%%%%%%%%%%%%%%%%%%%%%%%%%%%%%%%%%%%%%%%%%%%%%%%%%%%%%%%%%%%%%%%%%%%%%%%%%%%%%%%%%%%%%%%%%%%%%%%
%TCIDATA{OutputFilter=LATEX.DLL}
%TCIDATA{Version=5.00.0.2552}
%TCIDATA{<META NAME="SaveForMode" CONTENT="1">}
%TCIDATA{Created=Monday, May 30, 2005 10:44:34}
%TCIDATA{LastRevised=Sunday, June 05, 2005 11:19:18}
%TCIDATA{<META NAME="GraphicsSave" CONTENT="32">}
%TCIDATA{<META NAME="DocumentShell" CONTENT="Standard LaTeX\Blank - Standard LaTeX Article">}
%TCIDATA{CSTFile=40 LaTeX article.cst}

\setlength{\textheight}{8in}
\setlength{\textwidth}{6in}
\setlength{\oddsidemargin}{0mm}
\setlength{\evensidemargin}{0mm}
\setlength{\marginparwidth}{0mm}
\setlength{\marginparsep}{0mm}
\newtheorem{theorem}{Teorema}
\newtheorem{acknowledgement}[theorem]{Acknowledgement}
\newtheorem{algorithm}[theorem]{Algorithm}
\newtheorem{axiom}[theorem]{Axioma}
\newtheorem{case}[theorem]{Case}
\newtheorem{claim}[theorem]{Claim}
\newtheorem{conclusion}[theorem]{Concluzie}
\newtheorem{condition}[theorem]{Condition}
\newtheorem{conjecture}[theorem]{Conjecture}
\newtheorem{corollary}[theorem]{Corolar}
\newtheorem{criterion}[theorem]{Criteriu}
\newtheorem{definition}[theorem]{Definitie}
\newtheorem{example}[theorem]{Exemplu}
\newtheorem{exercise}[theorem]{Exercitiu}
\newtheorem{lemma}[theorem]{Lemma}
\newtheorem{notation}[theorem]{Notatie}
\newtheorem{problem}[theorem]{Problema}
\newtheorem{proposition}[theorem]{Propositie}
\newtheorem{remark}[theorem]{Remarka}
\newtheorem{solution}[theorem]{Solutie}
\newtheorem{summary}[theorem]{Summary}
\newenvironment{proof}[1][Proof]{\noindent\textbf{#1.} }{\ \rule{0.5em}{0.5em}}
\input{tcilatex}

\begin{document}

\title{Testul Z}
\author{Copyright 2005 Gabriel Dimitriu}
\maketitle

\begin{exercise}
Fie X\symbol{126}N($\mu $,81)
\end{exercise}

Sa se verifica la pragul de semnificatie $\alpha =0.05$ ipoteza $H_{0}:\mu
=840$ fata de $H_{1}:\mu <840$ folosindu-se o selectie de volum n=25 pentru
care $\overline{x}=838$.

\textbf{Rezolvare:}

Deoarece avem $H_{1}:\mu <m_{0}$ si dispersia $\sigma ^{2}=81$ vom aplica
testul Z unitalteral stinga pentru medie.

Vom calcula valoarea critica pentru acest tip de test

\[
\times _{c}=m_{0}+\frac{\sigma }{\sqrt{n}}u_{\alpha }=840+\frac{9}{5}%
(-1.64)=837,048 
\]

Deoarece $\overline{x}>x_{c}$ accept ipoteza $H_{0}$.

\begin{exercise}
Fie X\symbol{126}N($\mu $,144)
\end{exercise}

i) Sa se verifice la pragul de semnificatie $\alpha =0.01$ ipoteza $%
H_{0}:\mu =74$ fata de $H_{1}:\mu >74$ pe baza unei selectii de volum n=36
pentru care $\overline{x}=76$

ii) Sa se calculeze $\pi (78)$.

\textbf{Rezolvare:}

i) Deoarece avem $H_{1}:\mu >m_{0}$ si dispersia $\sigma ^{2}=144$ vom
aplica testul Z unilateral stinga pentru medie.

Vom calcula valoarea critica pentru acest tip de test

\[
x_{c}=m_{0}+\frac{\sigma }{\sqrt{n}}u_{1-\alpha }=74+\frac{12}{6}2.33=78.66 
\]

Deoarece $\overline{x}<x_{c}$ $(76<78.66)$ voi accepta ipoteza $H_{0}$.

ii)Puterea testului este data de formula

\[
\pi (m_{1})=\Phi \left( u_{\alpha }+\frac{m_{1}-m_{0}}{\sigma /\sqrt{n}}%
\right) 
\]

Inlocuind avem:

\[
\pi (78)=\Phi (-2.33+\frac{78-74}{12/6})=\Phi (-0.33)=1-\Phi
(0.33)=1-0.6293=0.3707 
\]

\begin{exercise}
\bigskip
\end{exercise}

O fabrica de televizoare afirma ca pentru atingerea unui anumit nivel (grad)
de stralucire sunt necesario 320 microamperi. O selectie de 28 astfel de
televizoare a dat $\overline{x}=331$. Amerajul necesar pentru a atinge
stralucirea dorita este o variabila $\ N(\mu ,144)$.

i)La pragul de semnificatie $\alpha =0.05$ sa se verifice ipoteza $H_{0}:\mu
=320$ fata de alternativa corespunzatoare.

ii)Daca $\mu =330$ care este probabilitatea erorii de ordinul doi ?

\textbf{Rezolvare:}

i) Deoarece ipoteza $H_{1}:\mu \neq 320$ vom folosi testul Z bilateral.

Vom calcula \ regiunea critica pentru $\sigma =12$.

\[
\left\vert \frac{\overline{x}-m_{0}}{\sigma /\sqrt{n}}\right\vert
=\left\vert \frac{331-320}{12/5.29}\right\vert =4.849 
\]

Pe care o comparam cu cuantila normala $u_{1-\alpha /2}=u_{0.975}=1.96$

Deoarece 4.849\TEXTsymbol{>}1.96 vom respinge ipoteza H$_{0}$.

ii)Probabilitatea erorii de ordin 2 este:

\begin{eqnarray*}
P(u &\in &W|H_{1})=\beta \\
P(|u| &<&u_{tab}|m_{1})=P(u<u_{tab})+P(u>u_{tab})=\Phi (u_{tab})+1-\Phi
(-u_{tab})=2\Phi (u_{tab})
\end{eqnarray*}

Deci

\[
\beta =2\ast \Phi (1.96)=1.95??? 
\]

\begin{exercise}
Recolta unei anumite cereale este o variabila X\symbol{126}N($\mu $,0.25)
\end{exercise}

Daca de \ pe 100ha s-a obtinut o recolta medie $\overline{x}=2050Kg/ha$ sa
se verifice, la pragul de semnificatie $\alpha =0.05$, ipoteza $H_{0}:\mu
=2000$ fata de $H_{1}:\mu >2000$.

\textbf{Rezolvare:}

Deoarece avem $H_{1}:\mu >m_{0}$ si dispersia $\sigma ^{2}=0.25$ vom aplica
testul Z unilateral stinga pentru medie.

Vom calcula valoarea critica pentru acest tip de test

\[
x_{c}=m_{0}+\frac{\sigma }{\sqrt{n}}u_{1-\alpha }=2000+\frac{0.5}{10}%
2.57=2000.128 
\]

Deoarece $\overline{x}=2050>x_{c}=2000.128$ vom accepta ipoteza $H_{0}.$

\begin{exercise}
\bigskip 
\end{exercise}

O selectie de 16 loturi de caprolactana cristalizata,tip A are continutul de
baze volatile $\overline{x}=0.22$ miliechivalenti pe Kg. Presupunem ca
continutul de baze volatile este o variabila aleatoare avind repartitia $%
N(\mu ,(0.08)^{2})$.

i)La pragul de semnificatie $\alpha =0.01$ sa se verifice ipoteza $H_{0}:\mu
=0.20$ fata de $H_{1}:\mu >0.20$.

ii)Care este puterea testului pentru $\mu _{1}=0.21$ ?

\textbf{Rezolvare:}

Deoarece avem $H_{1}:\mu >m_{0}$ si dispersia $\sigma ^{2}=0.08^{2}$ vom
aplica testul Z unilateral stinga pentru medie.

Vom calcula valoarea critica pentru acest tip de test

\[
x_{c}=m_{0}+\frac{\sigma }{\sqrt{n}}u_{1-\alpha }=0.20+\frac{0.08}{4}%
2.225=0.2445 
\]

Deoarece $\overline{x}<x_{c}$ $(0.22<0.2445)$ voi accepta ipoteza $H_{0}$.

ii)Puterea testului este data de formula

\[
\pi (m_{1})=\Phi \left( u_{\alpha }+\frac{m_{1}-m_{0}}{\sigma /\sqrt{n}}%
\right) 
\]

Inlocuind avem:

\[
\pi (0.21)=\Phi (-2.225+\frac{0.21-0.2}{0.08/4})=\Phi (-1.725)=1-\Phi
(1.725)=1-0.9577=0.04226 
\]

\begin{exercise}
\bigskip 
\end{exercise}

Experienta anterioara arata ca durabilitatea unei anvelope auto poate fi
considerata ca o variabila $N(30000Km,(800Km)^{2})$. Se face o schimbare a
procesului de productie. O selectie de 100 anvelope are $\overline{x}=29000Km
$. Pe baza acestei selectii si la un prag de semnificatie $\alpha =0.05$
putem spune ca noua metoda conduce la scaderea durabilitatii anvelopelor ?

\textbf{Rezolvare:}

Acesta problema se poate pune in cadrul ipotezelor astfel:

\begin{eqnarray*}
H_{0} &:&\mu =30000 \\
H_{1} &:&\mu <30000
\end{eqnarray*}

Deoarece avem $H_{1}:\mu <m_{0}$ si dispersia $\sigma ^{2}=800^{2}$ vom
aplica testul Z unitalteral stinga pentru medie.

Vom calcula valoarea critica pentru acest tip de test

\[
\times _{c}=m_{0}+\frac{\sigma }{\sqrt{n}}u_{\alpha }=30000+\frac{800}{10}%
(-1.645)=29868.4 
\]

Deoarece $\overline{x}=29000<x_{c}=29868.4$ resping ipoteza $H_{0}$ deci
intr-adevar noua metoda conduce la scaderea durabilitatii anvelopelor.

\begin{exercise}
\bigskip 
\end{exercise}

Durata de functionare a unui tip oarecare de bec electric de 100 W, poate fi
considerata o variabila aleatoare $X\symbol{126}N(1500,(200)^{2})$. O
selectie de 25 astfel de becuri da o durata medie de functionare de 1380 de
ore. La pragul de semnificatie $\alpha =0.01$ sa se verifice ipoteza $%
H_{0}:\mu =1500$ fata de alternativa $H_{1}:\mu <1500$.

\textbf{Rezolvare:}

Deoarece avem $H_{1}:\mu <m_{0}$ si dispersia $\sigma ^{2}=200^{2}$ vom
aplica testul Z unitalteral stinga pentru medie.

Vom calcula valoarea critica pentru acest tip de test

\[
\times _{c}=m_{0}+\frac{\sigma }{\sqrt{n}}u_{\alpha }=1500+\frac{200}{5}%
(-2.325)=1407 
\]

Deoarece $\overline{x}=1380<x_{c}=1407$ resping ipoteza $H_{0}$.

\begin{exercise}
\bigskip 
\end{exercise}

Masa medie a lucuitorilor unui oras poate fi considerata ca o variabila
aleatoare $X\symbol{126}N(70,5^{2})$. O selectie de 100 locuitori ai
orajului, cu domiciliul in zona parcurilor, este gasita ca avind o masa de
69Kg.

i)Acest rezultat indica faptul ca locuitorii avind domiciliul in zona
parcurior au o masa mai mica decit a celorlalti locuitori la pragul de
semnificatie $\alpha =0.05$ ?

ii)Care este puteea tetului pentru $\mu _{1}=68$ ?

\textbf{Rezolvare:}

i)Acest test poate fi vazut in cadrul ipotezelor statistice astfel:

\begin{eqnarray*}
H_{0} &:&\mu =70 \\
H_{1} &:&\mu <70
\end{eqnarray*}

Deoarece avem $H_{1}:\mu <m_{0}$ si dispersia $\sigma ^{2}=5^{2}$ vom aplica
testul Z unitalteral stinga pentru medie.

Vom calcula valoarea critica pentru acest tip de test

\[
\times _{c}=m_{0}+\frac{\sigma }{\sqrt{n}}u_{\alpha }=70+\frac{5}{10}%
(-1.645)=69.1775 
\]

Deoarece $\overline{x}=69<x_{c}=69.1775$ resping ipoteza $H_{0}$ deci
locuitorii avind domiciliul in zona parcurilor au o masa mai mica decit a
celorlalti locuitori.

ii)Puterea testului este data de formula

\[
\pi (m_{1})=\Phi \left( u_{\alpha }+\frac{m_{1}-m_{0}}{\sigma /\sqrt{n}}%
\right) 
\]

Inlocuind avem:

\[
\pi (68)=\Phi \left( -1.645+\frac{68-70}{5/10}\right) =\Phi (-5.645)=1-\Phi
(5.645)=0!! 
\]

\end{document}
