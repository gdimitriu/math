
\documentclass[a4paper]{article}
%%%%%%%%%%%%%%%%%%%%%%%%%%%%%%%%%%%%%%%%%%%%%%%%%%%%%%%%%%%%%%%%%%%%%%%%%%%%%%%%%%%%%%%%%%%%%%%%%%%%%%%%%%%%%%%%%%%%%%%%%%%%%%%%%%%%%%%%%%%%%%%%%%%%%%%%%%%%%%%%%%%%%%%%%%%%%%%%%%%%%%%%%%%%%%%%%%%%%%%%%%%%%%%%%%%%%%%%%%%%%%%%%%%%%%%%%%%%%%%%%%%%%%%%%%%%
%TCIDATA{OutputFilter=LATEX.DLL}
%TCIDATA{Version=5.00.0.2552}
%TCIDATA{<META NAME="SaveForMode" CONTENT="1">}
%TCIDATA{Created=Wednesday, July 13, 2005 11:40:27}
%TCIDATA{LastRevised=Wednesday, July 13, 2005 19:02:28}
%TCIDATA{<META NAME="GraphicsSave" CONTENT="32">}
%TCIDATA{<META NAME="DocumentShell" CONTENT="Standard LaTeX\Blank - Standard LaTeX Article">}
%TCIDATA{Language=American English}
%TCIDATA{CSTFile=40 LaTeX article.cst}

\setlength{\textheight}{8in}
\setlength{\textwidth}{6in}
\setlength{\oddsidemargin}{0mm}
\setlength{\evensidemargin}{0mm}
\setlength{\marginparwidth}{0mm}
\setlength{\marginparsep}{0mm}
\newtheorem{theorem}{Teorema}
\newtheorem{acknowledgement}[theorem]{Acknowledgement}
\newtheorem{algorithm}[theorem]{Algorithm}
\newtheorem{axiom}[theorem]{Axiom}
\newtheorem{case}[theorem]{Case}
\newtheorem{claim}[theorem]{Claim}
\newtheorem{conclusion}[theorem]{Conclusion}
\newtheorem{condition}[theorem]{Condition}
\newtheorem{conjecture}[theorem]{Conjecture}
\newtheorem{corollary}[theorem]{Corollary}
\newtheorem{criterion}[theorem]{Criterion}
\newtheorem{definition}[theorem]{Definition}
\newtheorem{example}[theorem]{Example}
\newtheorem{exercise}[theorem]{Exercise}
\newtheorem{lemma}[theorem]{Lemma}
\newtheorem{notation}[theorem]{Notation}
\newtheorem{problem}[theorem]{Problem}
\newtheorem{proposition}[theorem]{Proposition}
\newtheorem{remark}[theorem]{Remark}
\newtheorem{solution}[theorem]{Solution}
\newtheorem{summary}[theorem]{Summary}
\newenvironment{proof}[1][Proof]{\noindent\textbf{Proof} }{\ \rule{0.5em}{0.5em}}
\input{tcilatex}

\begin{document}

\date{}
\title{Computing an plotting the evolute of a curve}
\author{}
\maketitle

This project was done for Differential Geometry Course taught by Prof.
Gabriel Pripoaie at students from second\ year of University Spiru-Haret,
Faculty of Mathematics and Computer Science.

\section{Introduction}

\begin{definition}
Let $c:I\rightarrow E_{n}$ be a parametrized curve. If the following vectors
are linear independents $c^{(1)}(t),c^{(2)}(t),...,c^{(n-1)}(t)$ $\forall
t\in I$ we consider $c$ is in general position.
\end{definition}

\begin{proposition}
Let $c:I\rightarrow E_{n}$ be a curve in general position and let \{$%
e_{1},...,e_{n}\}$ be the Frenet mark associated to the curve. We have the
following relations:%
\begin{eqnarray*}
e_{i}^{\prime }(t) &=&\sum_{j=1}^{n}a_{ij}(t)e_{j}(t) \\
a_{ij}(t)+a_{ji}(t) &=&0,\forall i,j\in \{1,...,n\} \\
a_{ij}(t) &=&0,if\,\,\,j>i+1
\end{eqnarray*}
\end{proposition}

\begin{definition}
Let $c:I\rightarrow E_{n}$ be a curve in general position and let \{$%
e_{1},...,e_{n}\}$ be the Frenet mark associated to the curve. We call the
following function $K_{i}:I\rightarrow R$ the curvatures of $c$ curve in the
point $c(t)$ and they are defined by the following relation%
\begin{eqnarray*}
K_{i}(t) &=&\frac{a_{ii+1}(t)}{\left\Vert c^{\prime }(t)\right\Vert },i\in
\{1,...,n-1\} \\
a_{ii+1}(t) &=&\left\langle e_{i}^{\prime }(t),e_{i+1}^{\prime
}(t)\right\rangle 
\end{eqnarray*}
\end{definition}

\begin{definition}
Let consider a curve $c:I\rightarrow E_{2}$ and let be $t_{0}\in I$ so that $%
c^{\prime }(t_{0})\neq 0$ and $K_{1}(t_{0})\neq 0$. If a circle has three
confronted points in $c(t_{0})$ with the image of the curve $c$ then it is
called osculator circle of the curve in the point $c(t_{0})$.
\end{definition}

\begin{definition}
Let $c:t\in I\rightarrow c(t)=(x(t),y(t))\in E_{2}$ a regulated curve with $%
K_{1}(t)\neq 0,\forall t\in I$.The osculator circle of the curve $c$ in any
point $c(t)$ has the center $\Omega =(X(t),Y(t))$. The curve $\Gamma :t\in
I\rightarrow \Gamma (t)=(X(t),Y(t))\in E_{2}$ is called evolute of the curve 
$c$.Where X(t) and Y(t) are defined by%
\begin{eqnarray*}
X(t) &=&x(t)-\frac{y^{\prime }(t)(x^{\prime 2}(t)+y^{\prime 2}(t))}{%
x^{\prime }(t)y^{(2)}(t)-x^{(2)}(t)y^{\prime }(t)} \\
Y(t) &=&y(t)+\frac{x^{\prime }(t)(x^{\prime 2}(t)+y^{\prime 2}(t))}{%
x^{\prime }(t)y^{(2)}(t)-x^{(2)}(t)y^{\prime }(t)}
\end{eqnarray*}
\end{definition}

\section{Implementation}

The implementation is done in MAPLE\ because I have there differentiation,
solving and plotting of data.

\begin{algorithm}
\end{algorithm}

\begin{itemize}
\item compute $x^{\prime }(t)y^{(2)}(t)-x^{(2)}(t)y^{\prime }(t)$ and find
it's solutions. This points will be eliminated from the plot, because the
function isn't define in those points.

\item compute $K_{1}(t)=a_{ii+1}(t)/\left\Vert c^{\prime }(t)\right\Vert $.

\item compute $c^{\prime }(t)=0$ and find it's solutions. This points will
be also eliminated from the plot. 

\item compute analytical $X(t)$ and $Y(t)$

\item plot with blue the evolute and with red the curve

\item mark the singularity points where the curve is zero
\end{itemize}

\begin{exercise}
Compute and plot the evolute for cicloide.
\end{exercise}

\textbf{Solution:}

The cicloide is given by the following curve%
\[
c:I\subset R\rightarrow E_{2},a>0\,\,where\,c(t)=(a\ast (t-\sin (t)),a\ast
(1-\cos (t))
\]

Let be $a=2$ and $I=[-2\pi ,5\pi ]$. 

In this case we have $K=4(\cos (t)-1)\left( \left\vert -2+2\cos
(t)\right\vert ^{2}+4\left\vert \sin (t)\right\vert ^{2}\right) ^{-3/2}$ and
the parametric coordinates of osculator circle are $(2t+2\sin (t),-2+2\cos
(t))$.

\FRAME{dtbphF}{3.6646in}{3.5724in}{0pt}{}{}{Figure}{\special{language
"Scientific Word";type "GRAPHIC";maintain-aspect-ratio TRUE;display
"USEDEF";valid_file "T";width 3.6646in;height 3.5724in;depth
0pt;original-width 4.0714in;original-height 3.9684in;cropleft "0";croptop
"1";cropright "1";cropbottom "0";tempfilename
'evolute.bmp';tempfile-properties "XPR";}}

\end{document}
