
\documentclass[a4paper]{article}
%%%%%%%%%%%%%%%%%%%%%%%%%%%%%%%%%%%%%%%%%%%%%%%%%%%%%%%%%%%%%%%%%%%%%%%%%%%%%%%%%%%%%%%%%%%%%%%%%%%%%%%%%%%%%%%%%%%%%%%%%%%%%%%%%%%%%%%%%%%%%%%%%%%%%%%%%%%%%%%%%%%%%%%%%%%%%%%%%%%%%%%%%%%%%%%%%%%%%%%%%%%%%%%%%%%%%%%%%%%%%%%%%%%%%%%%%%%%%%%%%%%%%%%%%%%%
\usepackage{amsmath}
\usepackage{graphicx}

\setcounter{MaxMatrixCols}{10}
%TCIDATA{OutputFilter=LATEX.DLL}
%TCIDATA{Version=5.00.0.2552}
%TCIDATA{<META NAME="SaveForMode" CONTENT="1">}
%TCIDATA{Created=Wednesday, July 20, 2005 14:14:42}
%TCIDATA{LastRevised=Friday, July 22, 2005 19:28:10}
%TCIDATA{<META NAME="GraphicsSave" CONTENT="32">}
%TCIDATA{<META NAME="DocumentShell" CONTENT="Standard LaTeX\Blank - Standard LaTeX Article">}
%TCIDATA{Language=American English}
%TCIDATA{CSTFile=40 LaTeX article.cst}

\setlength{\textheight}{8in}
\setlength{\textwidth}{6in}
\setlength{\oddsidemargin}{0mm}
\setlength{\evensidemargin}{0mm}
\setlength{\marginparwidth}{0mm}
\setlength{\marginparsep}{0mm}
\newtheorem{theorem}{Teorema}
\newtheorem{acknowledgement}[theorem]{Acknowledgement}
\newtheorem{algorithm}[theorem]{Algorithm}
\newtheorem{axiom}[theorem]{Axiom}
\newtheorem{case}[theorem]{Case}
\newtheorem{claim}[theorem]{Claim}
\newtheorem{conclusion}[theorem]{Conclusion}
\newtheorem{condition}[theorem]{Condition}
\newtheorem{conjecture}[theorem]{Conjecture}
\newtheorem{corollary}[theorem]{Corollary}
\newtheorem{criterion}[theorem]{Criterion}
\newtheorem{definition}[theorem]{Definition}
\newtheorem{example}[theorem]{Example}
\newtheorem{exercise}[theorem]{Exercise}
\newtheorem{lemma}[theorem]{Lemma}
\newtheorem{notation}[theorem]{Notation}
\newtheorem{problem}[theorem]{Problem}
\newtheorem{proposition}[theorem]{Proposition}
\newtheorem{remark}[theorem]{Remark}
\newtheorem{solution}[theorem]{Solution}
\newtheorem{summary}[theorem]{Summary}
\newenvironment{proof}[1][Proof]{\noindent\textbf{Proof} }{\ \rule{0.5em}{0.5em}}
\input{tcilatex}

\begin{document}

\title{Computing and Plotting the Gaussian Mark}
\date{}
\author{}
\maketitle

This project was done for Differential Geometry Course taught by Professor
Gabriel Pripoaie at students from second year of University Spiru-Haret,
Faculty of Mathematics and Computer Science.

\section{Introduction}

Let $u=\overset{o}{u}\subseteq R^{n}$ and $f:u\rightarrow E^{n+1}$ be a
differentiable function \ with $x\in u$ ,with $%
f(x)=(f^{1}(x),f^{2}(x),...,f^{n+1}(x))$.

\begin{definition}
The function $f:u\subset R^{n}\rightarrow R^{n+1}$ is called immersion if $%
rangJ_{f}(x)$ is $n$ for each $x\in u$.
\end{definition}

\begin{definition}
The function f is called parametrized surface if f is immersion in any
domain point.
\end{definition}

\begin{notation}
Let note 
\begin{equation*}
df_{x}(e_{i})=f_{x^{i}}(x)=\frac{\partial f^{\alpha }}{\partial x^{i}}(x)%
\overline{e_{\alpha }}
\end{equation*}
\end{notation}

\begin{notation}
Let $T_{f(x)}R^{n+1}$ be $R^{n+1}$ and $df_{x}(T_{x}R^{n})=T_{f(x)}f=sp\{%
\frac{\partial f}{\partial x^{1}}(x),...,\frac{\partial f}{\partial x^{n}}%
(x)\}$
\end{notation}

\begin{definition}
Let $f:u\rightarrow E^{n+1}$ be a parametrized hypersurface and $%
X:u\rightarrow E^{n+1}$ be a differential application we call X field of
vectors along hypersurface f.
\end{definition}

\begin{definition}
Let $X:u\rightarrow E^{n+1}$ be a field of vectors along hypersurface f and
if $\forall x\in u,X(x)\in T_{f(x)}f$ then X is tangent.
\end{definition}

\begin{definition}
Let $X:u\rightarrow E^{n+1}$ be a field of vectors along hypersurface f and
if $\forall x\in u,X(x)\bot T_{f(x)}f$ then X is normal to f.
\end{definition}

\begin{definition}
$N(x)=\frac{f_{x^{1}}(x)\times f_{x^{2}}(x)\times ...f_{x^{n}}(x)}{%
\left\Vert f_{x^{1}}(x)\times f_{x^{2}}(x)\times ...f_{x^{n}}(x)\right\Vert }
$ is called unity normal to f.
\end{definition}

\begin{definition}
Let $f:u\rightarrow E^{n+1}$ be a parametrized hypersurface then $%
\{f_{x^{1}},...,f_{x^{n}},N\}$ we call Gaussian mark associated to f.
\end{definition}

\begin{definition}
Let $%
I_{x}(v,w)=<df_{x}(v),df_{x}(w)>=g_{x}(df_{x}(v),df_{x}(w))=v^{i}w^{j}g_{x}(f_{x^{i}},f_{x^{j}}) 
$ be the first fundamental form of hypersurface f in x, and $I_{x}=g_{x}$ be
the first fundamental form associated to hypersurface f.
\end{definition}

\begin{definition}
Let $f(X,Y)=<df_{p}(X),df_{p}(Y)>$ with $X,Y\in T_{f(p)}f$ , $\forall p\in u$
be the second fundamental associated to hypersurface f,$%
h_{ij}(p)=<f_{x^{i}x^{j}}(p),N(p)>.$
\end{definition}

\section{Implementation}

This is implemented in MAPLE because I have there differential, solving and
plotting of data.

\begin{algorithm}
\ \ 

\begin{itemize}
\item compute the Jacobian and if the rank of the Jacobian is zero then we
have immersion so we can continue else we return

\item find the points where $<f_{x^{1}}(x)\times f_{x^{2}}(x)\times
...f_{x^{n}}(x),f_{x}(x)>\neq 0$ because in these points we don't have
normal to surface and print these points.

\item compute N

\item compute and print the Gaussian Mark $\{f_{x^{1}},...,f_{x^{n}},N\}$

\item compute and print first and second fundamental forms.

\item make the translation of the Gaussian Mark to the surface by symbolic
adding the hypersurface.
\end{itemize}
\end{algorithm}

\begin{exercise}
Find the first and second fundamental forms and plot the curve with the
Gaussian Mark. The curve is given by the following parametrized relation for 
$\forall x\in \lbrack -\pi /2,\pi /2],\forall y\in \lbrack -2,0]$%
\begin{equation*}
f(x,y)=(2\ast \cos (x)\ast \cos (y),2\ast \cos (x)\ast \sin (y),2\ast \sin
(x))
\end{equation*}
\end{exercise}

In the following point we don't have normal to surface (1.570796327,y).

The first and second fundamental form are:

\begin{center}
$g(x,y)=%
\begin{bmatrix}
4 & 0 \\ 
0 & 4\cos (x)^{2}%
\end{bmatrix}%
,h(x,y)=%
\begin{bmatrix}
\frac{2\cos (x)}{|\cos (x)|} & 0 \\ 
0 & 2|\cos (x)|\cos (x)%
\end{bmatrix}%
$
\end{center}

The Gaussian Mark is:

\begin{equation*}
\begin{array}{c}
f_{x^{1}}(x,y)=(-2sin(x)\cos (y),-2\sin (x)\sin (y),2\cos (x)) \\ 
f_{x^{2}}(x,y)=\,(-2\cos (x)\sin (y),2\cos (x)\cos (y),0) \\ 
N(x,y)=\left( -\frac{\cos (x)^{2}\cos (y)}{|\cos (x)|},-\frac{\cos
(x)^{2}\sin (y)}{|\cos (x)},-\frac{\sin (x)\cos (x)}{|\cos (x)|}\right) 
\end{array}%
\end{equation*}

\begin{center}
\FRAME{dtbpFX}{3.2486in}{2.655in}{0pt}{}{}{sph.bmp}{\special{language
"Scientific Word";type "GRAPHIC";maintain-aspect-ratio TRUE;display
"FULL";valid_file "F";width 3.2486in;height 2.655in;depth 0pt;original-width
4.9406in;original-height 4.0315in;cropleft "0";croptop "1";cropright
"1";cropbottom "0";filename 'sph.bmp';file-properties "XNPEU";}}
\end{center}

\end{document}
